\chapter{The Schwinger Variational Method}
\textbf{Note}
Here you will see how to cite references and organize your chapter.  
Feel free to use this structure as a template for your own writing.
% You can cite like this\cite{RefA}, or like this \cite{RefA,RefB,RefC}.

The Schwinger variational principle provides one of the most powerful
formulations of scattering theory. Instead of solving the full
Lippmann–Schwinger equation directly, the method constructs a functional
whose stationary value yields the exact transition amplitude
\cite{schwinger1951, joachain1975}.

\section{Application to Proton–Hydrogen Collisions}

For proton–hydrogen excitation, the interaction potential is dominated by the
Coulomb term and the coupling between \(1s\) and excited \(nl\) states of the
hydrogen electron. Variational treatments have been shown to reproduce
experimental cross sections with high accuracy, even at intermediate energies
\cite{ohrn1982, mccarthy2001}.
