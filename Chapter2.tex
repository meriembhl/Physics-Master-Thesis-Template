\chapter{Theoretical Background}

In this chapter, we introduce the fundamental theoretical tools necessary for analyzing
atomic scattering processes. The goal is to provide a clear mathematical and physical
foundation before applying the Schwinger variational principle in later chapters.

\section{The Hydrogen Atom}
The hydrogen atom plays a central role in quantum mechanics due to its exact analytical
solutions. The time-independent Schrödinger equation is given by:

\begin{equation}
    \hat{H}\psi(\mathbf{r}) = E\psi(\mathbf{r}),
\end{equation}

where the Hamiltonian takes the form:

\begin{equation}
    \hat{H} = -\frac{\hbar^2}{2m}\nabla^2 - \frac{e^2}{4\pi\epsilon_0 r}.
\end{equation}

The energy levels are:

\begin{equation}
    E_n = -\frac{13.6\ \text{eV}}{n^2}, \qquad n = 1, 2, 3, \dots
\end{equation}

\section{Radial Wavefunctions}

The hydrogen radial wavefunctions \( R_{nl}(r) \) are expressed using associated Laguerre polynomials.
For example, the ground state \( (n=1,\; l=0) \) is:

\begin{equation}
    R_{10}(r) = 2\left( \frac{1}{a_0} \right)^{3/2} e^{-r/a_0},
\end{equation}

where \( a_0 \) is the Bohr radius.

