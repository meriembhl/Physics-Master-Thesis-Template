\chapter{Introduction to Theoretical Physics}

This chapter demonstrates how to include equations, tables, figures..
 Replace all examples with your own work.

\section{Background}
Theoretical physics bridges the gap between abstract mathematical models and experimental observations.
It allows researchers to predict phenomena before they are observed.

\section{Key Equations}

One of the most famous results in physics is Einstein's mass–energy equivalence:

\begin{equation}
E = mc^2
\label{eq:einstein}
\end{equation}

Another important principle is the time-dependent Schrödinger equation for a particle in a potential:

\begin{equation}
i \hbar \frac{\partial \psi(\mathbf{r},t)}{\partial t} = \hat{H}\, \psi(\mathbf{r},t)
\label{eq:schrodinger}
\end{equation}

% ---------------- Figure Example ----------------

\begin{figure}[h!]
\centering
\includegraphics[width=0.6\textwidth]{10-Figures/feynman.jpg} % replace path/name if needed
\caption{Feynman diagram example illustrating particle interactions.}
\label{fig:feynman}
\end{figure}

% ---------------- Table Example ----------------

\begin{table}[h!]
\begin{center}
\begin{tabular}{|c|c|c|}
\hline
Parameter & Value & Units \\ \hline
Mass & $1.67\times10^{-27}$ & kg \\ \hline
Charge & $1.60\times10^{-19}$ & C \\ \hline
Energy & $1.5\times10^{-13}$ & J \\ \hline
\end{tabular}
\caption{Sample data for illustrative purposes.}
\label{tab:sample}
\end{center}
\end{table}

% ---------------- Summary ----------------
\section{Summary}
In this chapter, we introduced the fundamental concepts of theoretical physics, key equations, and illustrative examples.
Tables and figures provide a visual and numerical reference for concepts that will be developed in subsequent chapters.

% Placeholder note for students
% Replace this content with your own research background, equations, tables, and figures.
